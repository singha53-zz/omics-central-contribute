\documentclass[]{book}
\usepackage{lmodern}
\usepackage{amssymb,amsmath}
\usepackage{ifxetex,ifluatex}
\usepackage{fixltx2e} % provides \textsubscript
\ifnum 0\ifxetex 1\fi\ifluatex 1\fi=0 % if pdftex
  \usepackage[T1]{fontenc}
  \usepackage[utf8]{inputenc}
\else % if luatex or xelatex
  \ifxetex
    \usepackage{mathspec}
  \else
    \usepackage{fontspec}
  \fi
  \defaultfontfeatures{Ligatures=TeX,Scale=MatchLowercase}
\fi
% use upquote if available, for straight quotes in verbatim environments
\IfFileExists{upquote.sty}{\usepackage{upquote}}{}
% use microtype if available
\IfFileExists{microtype.sty}{%
\usepackage[]{microtype}
\UseMicrotypeSet[protrusion]{basicmath} % disable protrusion for tt fonts
}{}
\PassOptionsToPackage{hyphens}{url} % url is loaded by hyperref
\usepackage[unicode=true]{hyperref}
\hypersetup{
            pdftitle={Omics Central - contribute},
            pdfauthor={Amrit Singh},
            pdfborder={0 0 0},
            breaklinks=true}
\urlstyle{same}  % don't use monospace font for urls
\usepackage{natbib}
\bibliographystyle{apalike}
\usepackage{longtable,booktabs}
% Fix footnotes in tables (requires footnote package)
\IfFileExists{footnote.sty}{\usepackage{footnote}\makesavenoteenv{long table}}{}
\usepackage{graphicx,grffile}
\makeatletter
\def\maxwidth{\ifdim\Gin@nat@width>\linewidth\linewidth\else\Gin@nat@width\fi}
\def\maxheight{\ifdim\Gin@nat@height>\textheight\textheight\else\Gin@nat@height\fi}
\makeatother
% Scale images if necessary, so that they will not overflow the page
% margins by default, and it is still possible to overwrite the defaults
% using explicit options in \includegraphics[width, height, ...]{}
\setkeys{Gin}{width=\maxwidth,height=\maxheight,keepaspectratio}
\IfFileExists{parskip.sty}{%
\usepackage{parskip}
}{% else
\setlength{\parindent}{0pt}
\setlength{\parskip}{6pt plus 2pt minus 1pt}
}
\setlength{\emergencystretch}{3em}  % prevent overfull lines
\providecommand{\tightlist}{%
  \setlength{\itemsep}{0pt}\setlength{\parskip}{0pt}}
\setcounter{secnumdepth}{5}
% Redefines (sub)paragraphs to behave more like sections
\ifx\paragraph\undefined\else
\let\oldparagraph\paragraph
\renewcommand{\paragraph}[1]{\oldparagraph{#1}\mbox{}}
\fi
\ifx\subparagraph\undefined\else
\let\oldsubparagraph\subparagraph
\renewcommand{\subparagraph}[1]{\oldsubparagraph{#1}\mbox{}}
\fi

% set default figure placement to htbp
\makeatletter
\def\fps@figure{htbp}
\makeatother

\usepackage{booktabs}
\usepackage{amsthm}
\makeatletter
\def\thm@space@setup{%
  \thm@preskip=8pt plus 2pt minus 4pt
  \thm@postskip=\thm@preskip
}
\makeatother

\title{Omics Central - contribute}
\author{Amrit Singh}
\date{2020-01-08}

\begin{document}
\maketitle

{
\setcounter{tocdepth}{1}
\tableofcontents
}
\chapter*{Rationale}\label{rationale}
\addcontentsline{toc}{chapter}{Rationale}

This purpose of this book is to 1) provide a guide to self-host Omics
Central on one's own Amazon Web Services (AWS) account and 2) extending
omics central by contributing additional functionality (methods,
visualizations).

If users would like to add additional features or make changes to the
existing application they are welcome to submit pull requests to
(omics-central-front-end, omics-central-backend and
omics-central-docker) as well as to the documentation
(omics-central-learn) and implementation (omics-central-contribute).

\chapter{Introduction}\label{intro}

\chapter{Self-hosting}\label{self}

\section{AWS account setup}\label{aws-account-setup}

\section{Frontend}\label{frontend}

\section{Backend}\label{backend}

\subsection{Serverless}\label{serverless}

\subsection{ECS cluster}\label{ecs-cluster}

\chapter{Develop}\label{develop}

\section{Frontend}\label{frontend-1}

\subsection{CICD setup}\label{cicd-setup}

\subsection{React}\label{react}

\subsubsection{primer}\label{primer}

\subsubsection{pages}\label{pages}

\section{Backend}\label{backend-1}

\subsection{CICD setup}\label{cicd-setup-1}

\subsection{Serverless setup}\label{serverless-setup}

\subsubsection{App architecture}\label{app-architecture}

\subsubsection{Services}\label{services}

\subsection{Docker setup}\label{docker-setup}

\subsubsection{Dockerfile}\label{dockerfile}

\subsubsection{R scripts}\label{r-scripts}

\subsubsection{shell scripts}\label{shell-scripts}

\bibliography{book.bib,packages.bib}

\end{document}
